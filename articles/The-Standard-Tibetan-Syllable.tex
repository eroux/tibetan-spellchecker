\documentclass[%
a4paper,% paper size
pagesize,%
12pt,% text size
parskip=off,% inter-paragraph space, automatically sets paragraph indentation
bibliography=totoc,% bibliography in the table or contents
numbers=noenddot,%
DIV=12,% margin settings
twoside=semi,% right and left difference, but same margins
headings=normal%headings=medium,% small headings
]{scrartcl}

\title{The Standard Tibetan Syllable}
\subtitle{Composition rules and analysis}
\author{Hélios Hildt (Université Bordeaux Montaigne) \and Élie Roux} 
\date{\today}

\usepackage{tib-articles}
\usepackage[numbers,square]{natbib} % These need to be in this file!

\begin{document}

\maketitle

\begin{abstract}
Documents describing the composition of the standard tibetan syllable in a superficial way can be easily found in both Tibetan and Western language, but these miss many common features and rules. This document is an attempt to have a complete description on how a standard Tibetan syllable is built and how it can be analyzed, which we hope will provide a solid basis for NLP areas such as spell checking and collation.
\end{abstract}

\keywords{Tibetan, NLP, Spellchecking, Unicode}

\tableofcontents

\newpage

\section{Introduction}

This document presents our research on the different possibilities and constraints in the constructions of what we will call a \emph{standard tibetan syllable} (ཁྲིམས་མཐུན་གྱི་ཚེག་བར་, litteraly \emph{legal syllable}). Most of the rules comes from the orthograph reform of King རལ་པ་ཅན་ in the \textsc{IX}th century \textsc{C.E.}\footnote{this reform seems completely undocummented in Western language. For some documents predating the reform, a few from  the Dunuang corpus available at \url{https://github.com/eroux/Dunhuang-unicode} do, but not all.}, but Tibetans also consider standard tibetan syllables the regional or archaic words\footnote{respectively marked ཡུལ and རྙིང་ in dictionaries.} still commonly used, so we included them in our research.

We did not consider:

\begin{itemize}
\item translitterated sanskrit or chinese
\item proper nouns: even the very common person names ཀརྨ་ and པདྨ་ are not consider standard Tibetan by Tibetans
\item Dzongkha, obeying completely different rules
\item abreviations (བསྡུས་ཡིག་)
\end{itemize}

We studied from a theoretical point of view by looking at tibetan grammar books such as \cite{TsheshabGrammarTopics}; but also from a practical point of view by looking at dictionaries that Tibetans consider reliable (\cite{DorjeDagyig}, \cite{YisunTsikchen}, \cite{MonlamGrandDict} and \cite{DungkarEncyclopedia}) and Linguistic books on Tibetan (\cite{TournadreMST}).

We hope that this reserach will allow 

\section{Basic syllabic elements}

­\subsection{Traditional description}

A Standard Tibetan Syllable is traditionally described as being composed of a maximum of 8 elements :

\begin{description}
  \item[one root letter (མིང་གཞི་)] among the 30 standard tibetan letters (ཀ to ཨ)
  \item[zero or one (0/1) vowel (དབྱངས་):] {\tibetanfont ◌}ི {\tibetanfont ◌\kern 0.5mm}ུ {\tibetanfont ◌}ེ or {\tibetanfont ◌}ོ
  \item[0/1 superscript letter (མགོ་ཅན་):] ར ལ or ས
  \item[0/1 subscript letter (འདོགས་ཅན་):] ྱ ྲ or ླ
  \item[0/1 wasur (ཝ་ཟུར་):] ྭ
  \item[0/1 prefix (སྔོན་འཇུག་):] ག ད བ མ or འ
  \item[0/1 first suffix (རྗེས་འཇུག་):] ག ང ད ན བ མ འ ར ལ or ས
  \item[0/1 second suffix (ཡང་འཇུག་):] ས
\end{description}

Note that in traditional presentations, the wasur is counted among the subscript letters. We chose not to present things this way becauses it poses difficulties when the wasur is combined with another subscript letter, as in གྲྭ.

\subsection{Rare suffixes}

We will discuss in this section two types of suffixes that are not described in the traditional grammars.

The first is surprisingly common: འུ, acting as a diminituve and present in many words (བྱིའུ་, fledgeling). See Appendix 1 for a list of words containing this suffix.

The second type is constituted by suffixes appearing only in very few words:

\begin{description}
\item[བྲའོ་], a regional alternative spelling for བྲ་བོ་ (buckwheat)
\item[སླེའོ་], an archaic word designating the words of someone
\item[ཀྭའི་], an onomatopoeia
\end{description}

\cite{TournadreMST} also indicates འེ without giving any example. As we were able to find no standard tibetan word (only proper nouns) containing it in the dictionaries we consulted, we will consider that these are not valid.

These rare suffixes behave like regular suffixes, so we will consider them as such.

­\subsection{Affixed particles}

When a syllable has no suffix or suffix འ, it may carry an affixed particle\footnote{there is no word for this in Tibetan yet, we propose to call them ཚིག་སྔ་མར་བསྡུས་པའི་ཕྲད་} according to the context. When the affixed particle is applied ona syllable with suffix འ, this suffix is replaced by the particle.

The traditional list of affixed particles is འི འོ འང འམ ར and ས. They are implicitely presented as mutually exclusive, but our research shows that འིའོ (combination of འི and འོ), although very rare\footnote{it can be found around 200 time in the Derge Kangyur}, is considered standard Tibetan by our informants.

­\subsection{Archaic elements}

Before the orthograph reform of King རལ་པ་ཅན་, some elements of the syllable were different:

\begin{itemize}
\item a second suffix ད (called ད་དྲག་)
\item the vowel {\tibetanfont ◌}ི was often written {\tibetanfont ◌}ྀ (for unknown reasons)
\item the affixed particle ས was written འྀས
\item the affixed particle ར was written འར
\end{itemize}

Note that the last two differences allowed simple disambiguation between the ར and ས suffixes and their affixed particle equivalent; while in modern Tibetan, the orthograph is the same and the context must be used to disambiguate.

­\section{Rules of composition}

­\subsection{Basic rules}

The basic rules (or constraints) to compose a well-formed tibetan syllable can be found in many places (\cite{TournadreMST}, \cite{TsheshabGrammarTopics}). They are restrictions on the possible combinations of superscript + rootletter + subscript. Ignore wasurs (as they will be discussed in a separate section), the possible combinations (not counting the raw root letters) exposed in \cite{TsheshabGrammarTopics} can be summed up as follows:\label{mainstackdecomposition}

\begin{description}
\item[superscript ར] རྐ རྒ རྔ རྗ རྙ རྟ རྡ རྣ རྦ རྨ རྩ རྫ
\item[superscript ལ] ལྐ ལྒ ལྔ ལྕ ལྗ ལྟ ལྡ ལྤ ལྦ ལྷ
\item[superscript ས] སྐ སྒ སྔ སྙ སྟ སྡ སྣ སྤ སྦ སྨ སྩ
\item[subscript ྱ] ཀྱ ཁྱ གྱ པྱ ཕྱ བྱ མྱ
\item[subscript ྲ] ཀྲ ཁྲ གྲ ཏྲ ཐྲ དྲ པྲ ཕྲ བྲ སྲ ཧྲ
\item[subscript ླ] ཀླ གླ བླ ཟླ རླ སླ
\item[superscript ར + subscript ྱ] རྐྱ རྒྱ རྨྱ
\item[superscript ས + subscript ྱ] སྐྱ སྒྱ སྤྱ སྦྱ སྨྱ
\item[superscript ས + subscript ྲ] སྐྲ སྒྲ སྣྲ སྤྲ སྦྲ སྨྲ
\end{description} 

Our research in the main dictionaries\footnote{\cite{DorjeDagyig}, \cite{YisunTsikchen}, \cite{MonlamGrandDict} and \cite{DungkarEncyclopedia}.} showed that མྲ is also a standard possibility, only found in ཨ་མྲ (mango), so we consider it as an exception. \cite{TournadreMST} also lists ཤྲ, without giving any example, and we were not able to find any in the dictionaries we consulted except for proper nouns, so we do not consider it standard in this document.

­\subsection{More advanced rules}

An interesting feature of \cite{TsheshabGrammarTopics} that we were not able to find anywhere else is a more complete set of constaints including the prefix. The following list of all valid combinations of prefix + superscript + root letter + subscript can be easily built from the text, and constitutes an important contribution by this article:

\begin{itemize}
\item ཀ ཀྱ ཀྲ ཀླ དཀ དཀྱ དཀྲ དཀླ བཀ བཀྱ བཀྲ བཀླ རྐ རྐྱ ལྐ སྐ སྐྱ སྐྲ བརྐ བརྐྱ བསྐ བསྐྱ བསྐྲ
\item ཁ ཁྱ ཁྲ མཁ མཁྱ མཁྲ འཁ འཁྱ འཁྲ
\item ག གྱ གྲ གླ དག དགྱ དགྲ བག བགྱ བགྲ བགླ མག མགྱ མགྲ འག འགྱ འགྲ རྒ རྒྱ ལྒ སྒ སྒྱ སྒྲ བརྒ བརྒྱ བསྒ བསྒྱ བསྒྲ
\item ང དང མང རྔ ལྔ སྔ བརྔ བསྔ
\item ཅ གཅ བཅ ལྕ
\item ཆ མཆ འཆ
\item ཇ མཇ འཇ རྗ ལྗ བརྗ
\item ཉ གཉ མཉ རྙ སྙ བརྙ བསྙ
\item ཏ ཏྲ གཏ གཏྲ བཏ བཏྲ རྟ ལྟ སྟ བརྟ བལྟ བསྟ
\item ཐ ཐྲ མཐ འཐ
\item ད དྲ གད བད མད མདྲ འད འདྲ རྡ ལྡ སྡ བརྡ བལྡ བསྡ
\item ན གན མན རྣ སྣ སྣྲ བརྣ བསྣ
\item པ པྱ པྲ དཔ དཔྱ དཔྲ ལྤ སྤ སྤྱ སྤྲ
\item ཕ ཕྱ ཕྲ འཕ འཕྱ འཕྲ
\item བ བྱ བྲ བླ དབ དབྱ དབྲ འབ འབྱ འབྲ རྦ ལྦ སྦ སྦྱ སྦྲ
\item མ མྱ དམ དམྱ རྨ རྨྱ སྨ སྨྱ སྨྲ
\item ཙ གཙ བཙ རྩ སྩ བརྩ བསྩ
\item ཚ མཚ འཚ
\item ཛ མཛ འཛ རྫ བརྫ
\item ཝ
\item ཞ གཞ བཞ
\item ཟ ཟླ གཟ བཟ བཟླ
\item འ
\item ཡ གཡ
\item ར རླ བརླ
\item ལ
\item ཤ གཤ བཤ
\item ས སྲ སླ གས བས བསྲ བསླ
\item ཧ ཧྲ ལྷ
\item ཨ
\end{itemize}

\cite{TsheshabGrammarTopics} cites two exceptions of standard syllables not fitting in these constraints: བགླ and མདྲོན, to which we can add མྲ.

\subsection{Rules for the suffixes and second suffixes}

There seems to be no rule for suffixes (except འ discussed in next section): any suffix can appear after any combinaison of prefix, subscript, root letter, subscript, vowel and wasur.

The second suffix ས can only appear after the suffixes ག, ང, བ and མ. The archaid second suffix ད can only appear after suffix ན, ར or ལ.

­\subsection{Rules for the འ suffix}

The འ suffix has some rules that are quite simple but surprisingly undocumented. It is used only to disambiguate syllables. When reading syllables, Tibetans consider that when a syllable is composed of two consonnants with no other sign (no superscript, subscript or vowel), then the first is the root letter (see \ref{decompositionrules}). So for instance the syllable བཟ is not valid because it would be decomposed as root letter བ and suffix ཟ, which is an invialid suffix. If one wants to write prefix བ and root letter ཟ, then he must add the suffix འ: བཟའ. This suffix seems to have no other use. 

To sum up, the suffix འ can only appear in a syllable where there is exactly:

\begin{itemize}
\item one prefix
\item no superscript
\item no subscript
\item no vowel
\end{itemize}

In our research we were able to find only very few exceptions to this rule: 

\begin{itemize}
\item དམེའ archaic, found in many sources
\item བརྡའ mispelling for བརྡ
\item བརྟའ found only in \cite{KhartoTenses} (not considered very reliable by our Tibetan informants)
\end{itemize}

\subsubsection*{Note on the use of the འ suffix in archaic Tibetan}

In archaic Tibetan, the role of the འ suffix is quite unclear\footnote{\cite{TournadreMST} states that it appeared when a vowel is present and gives རིའ (mountain) as an example.}. In the Dunhuang corpus, many of its use correspond to the current one, but it's also used in other cases such as the very common རྒྱའ, པའ or ནའ. A careful study about this would be interesting but goes far beyond the scope of this document.

\subsection{Rules for the Wasur}

\subsubsection*{Traditional rules}

­The wasur is traditionally treated as a regular subscript letter, and the list of valid combinations is ཀྭ ཁྭ གྭ ཉྭ དྭ ཚྭ ཞྭ ཟྭ རྭ ལྭ ཤྭ ཧྭ གྲྭ རྩྭ ཕྱྭ (list given in \cite{TsheshabGrammarTopics}). \cite{TournadreMST} also cites ཏྭ, ཙྭ and སྭ.

\subsubsection*{Proposal}

These rules are unsatisfying for the purpose of spell checking for two reasons.

The first is that they are quite incomplete as they do not take སྟྭ (found in སྟྭ་རེ་) and དྲྭ (contraction of དྲ་བ་) into account.

The second is that there are so few standard tibetan syllables including a wasur compared to the many different possibilities, that it is much more satisfying to treat these words as exceptions. This is backed by the following arguments:

\begin{itemize}
\item the wasur seems to have no influence on prononciation, in all regions
\item when inventing new words, Tibetans don't seem to use wasurs\footnote{Not a single wasur appears in the 188 pages of \cite{TseringIT} for instance.}
\item tibetan grammarians seem a little embarassed with the role of the wasur and some consider it only as a disambiguation sign, see \ref{wasurdisambiguation}
\end{itemize}

Our opinion is thus to consider the wasur as an historical feature of the syllable, present in a set of syllable that can be listed. Our attempt at a list of standard Tibetan syllable with wasurs can be found in Annex 1.

­\section{Rules of decomposition}

The rules for finding the different elements when reading a Tibetan syllable are necessary to understand the roles of the འ suffix and the wasur. These simple rules can be found easily (\cite{TournadreMST}, etc.), but we propose here some very complete and systematic algorithm working for all standard tibetan syllable, plus a proposal for the ambiguous cases.

­\subsection{Rules for finding the main stack}

What we propose here is to find the main stack, meaning the combination of 1 root letter, 0/1 superscript letter, 0/1 subscript letter and 0/1 wasur. Once the main stack is found, it is trivial to find the root letter and the different elements. Note that these rules are valid for standard Tibetan syllables only, not for a random combination of Tibetan characters.

The following rules are meant to be applied in this particular order, and should be easy to implement for NLP purposes:

\subsubsection*{Simple cases}\label{decompositionrules}

\begin{description}
\item[1] if the syllable contains a subscript, superscript or wasur then the main stack is what contains it (ex: in དབྱངས the main stack is བྱ because it carries the subscript ྱ)
\item[2] if a consonant other than འ carries a vowel then it is the main stack (གཞི \rightarrow{} ཞི)
\end{description}

\subsubsection*{Explicit vowel above འ, no other explicit vowel, subscript nor superscript}

\begin{description}
\item[3] if a vowel is carried by འ and འ is the first letter then འ is the main stack (འོད \rightarrow{} འོ)
\item[4] if a vowel is carried by an འ which is not the first letter, the main stack is before the first འ with a vowel\footnote{This rule also works for archaic tibetan using འྀས.} (མའིའོ \rightarrow{} མ)
\end{description}

\subsubsection*{No explicit vowel, subscript nor superscript}

\begin{description}
\item[5] if the syllable has more than three consonnant and ends with འང or འམ, then the main stack is right before འང or འམ\footnote{This rule could be extended for archaic tibetan with འར.}
\item[6] if the syllable contains one consonnant, then this consonnant is the main stack
\item[7] if the syllable contains two consonnants, then the first is the main stack (བར \rightarrow{} བ)
\item[8] if the syllable contains three consonnants and the final consonnant is not ས, then the main stack is the second consonnant (བཟའ \rightarrow{} ཟ)
\item[9] if the syllable contains three consonnants and the final consonnant is ས, and the first consonnant cannot be a prefix (it is not ག, ད, བ, མ nor འ), then the main stack is the first consonnant (ཐགས \rightarrow{} ཐ)
\item[10] if the syllable contains three consonnants and the final consonnant is ས, and the first consonnant can be a prefix, and ས cannot be second prefix after the second consonnant (meaning the second consonnant is not ག, ང, བ nor མ), then the main stack is the first consonnant (གནས \rightarrow{} ག)
\item[11] if the syllable contains four consonnants, the main stack is the second consonnant (བཟབས \rightarrow{} ཟ)
\end{description}

­\subsubsection*{Ambiguous syllables}

When the syllable no explicit vowel, no superscript, no subscript, has three consonnants with a final ས, the first consonnant can be a prefix and the second consonnant is ག, ང, བ or མ, the case is ambiguous. There are 9 such cases in standard Tibetan མངས, མགས, དབས, དངས, དགས, དམས, བགས, འབས and འགས.

For each of these cases we have three possible structures:
\begin{enumerate}
\item ས is a second suffix (e.g. མངས can be decomposed as root leter མ, suffix ང, second suffix ས, hereafter noted མ|ངས)
\item ས is a suffix (e.g. མངས is root letter ང, prefix མ, suffix ས, noted མང|ས)
\item ས is an affixed particle (e.g. མངས is syllable མངའ transformed into མངས, root letter ང, prefix མ, suffix འ, affixed particle ས, noted མང|འ+ས)
\end{enumerate}

In order to decide where the main stack is, the only way is to take the decomposition with the highest probability according to our knowledge of the tibetan language.

Let's review the different forms for cases not involving ད, prefixing by * a form unattested in the dictionaries we consulted:

\begin{description}
\item[མབས:] མ|ངས, *མང|ས, མང|འ+ས \rightarrow{} \textbf{མ} is the main stack with reasonable probability
\item[འབས:] *འ|བས, *འབ|ས, འབ|འ+ས \rightarrow{} \textbf{བ}
\item[མགས:] *མ|གས, *མག|ས, *མག|འ+ས \rightarrow{} impossible to determine, but \textbf{མ} as main stack is more intuitive to our Tibetan informants
\item[བགས:] བ|གས, *བག|ས, *བག|འ+ས \rightarrow{} \textbf{བ}
\item[འགས:] *འ|གས, འག|ས, འག|འ+ས \rightarrow{} \textbf{ག}
\item[དབས:] *ད|བས, དབ|ས (archaic), དབ|འ+ས \rightarrow{} \textbf{བ}
\item[དགས:] *ད|གས, དག|ས, དག|འ+ས \rightarrow{} \textbf{ག}
\item[དངས:] ད|ངས (mispelling of དྭངས\footnote{attested in \cite{NegiDict} (vol. 6) and \cite{Illuminator}.}), *དང|ས, *དང|འ+ས \rightarrow{} \textbf{ད}
\item[དམས:] *ད|མས, དམ|ས, དམ|འ+ས \rightarrow{} \textbf{མ}
\end{description}

\subsubsection*{Wasur as a disambiguation mark}\label{wasurdisambiguation}

With these disambiguations in mind, we can now challenge a rule given in \cite{MugeGrammar} stating that the wasur is used for disambiguation. This rule states that in the ambiguous cases we described, the second consonnants is the main stack because if it is was the first, then the first would carry a wasur.

A first obvious objection is that among these 9 cases, only 4 involve a first letter that could carry a wasur (ད). So if the wasur was really made for disambiguation, then འ, མ and བ could carry wasurs, which is never found in any grammar or any existing tibetan syllable.

For these four cases, the only variants with wasurs that could be found in dictionaries are དྭངས and དྭགས (དྭམས and དྭབས are never attested). The rule works in three cases (དབས, དགས and དམས), but doesn't for དངས: the rule would indicate that it should be considered དང|ས while དང|ས doesn't exist and ད|ངས is attested (though as a mispelling).

So it seems that, although the wasur is helpful to quickly disambiguate དྭངས and དྭགས, its disambiguation purpose cannot be generalized.

\subsection{Rules for the complete decomposition}

Once the main stack is found, prefix, vowel and wasur are immediate to find. If a superscript or subscript is present, they can be immediately found with the rules exposed in \ref{mainstackdecomposition}.

For suffixes, all of them can be immediately classified between suffix, second suffix and affix particle, except ས and ར. 

The disambiguation of these two suffixes can be done many times at syllable level according to the existence of not of one version of the syllable. For example ཐས in which ས can only be affixed particle, because root letter ཐ + suffix ས is nowhere to be found in dictionnaries. 

But many common cases are not decidable without the context. The most common example would be བར, in which we cannot decide if ར is suffix or affixed particle. This is a very strong difficulty for some areas of NLP because this ambiguity makes phonetic transcription, particle inflection, etc. dependant of the context and thus difficult to achieve systematically. Note that this does not impact collation as both forms are strictly equivalent from a collation point of view.

\section{Conclusion}

During our research we have been able to find a few rules concerning the formation of a standard tibetan syllable that were never documented before in Western language, as well as interesting exceptions to common rules including:

\begin{itemize}
\item the constraints on prefix + superscript + root letter + subscript
\item documentation on the འུ suffix
\item clarification on the role of wasur and rules concerning it
\item formalization of the rules concerning the འ suffix
\item systematization of the rules to decompose a syllable, and how to deal with ambiguous cases
\end{itemize}

This research allowed us to create a spellchecker at syllable level for standard tibetan for the Hunspell library (used in many softwares including most free softwares, LibreOffice, the Adobe softwares, etc.)\footnote{The files are available for free on \url{https://github.com/eroux/hunspell-bo}.}. This spellchecker takes a few proper names into account (just like a spell checker in a Western language), although further research should be done for it to be complete in this area. This spell checker is of course very limitted as it works only at syllable level, not word, but we hope it will provide the foundation for a more complete one.

­\appendix

\section{List of syllables containing the འུ suffix or wasurs}

\subsection{List of syllables containing a wasur}

The following standard tibetan syllables can be found easily in many dictionaries:

ཀྭ་ ཀྭའི་ ཀྭས་ ཀྭན་ ཁྭ་ ཁྭངས་ གྭ་ གྲྭ་ ཚྭབ་ ཉྭ་ ཏྭོན་ སྟྭ་ དྭ་ དྭོ་ དྭང་ དྭངས་ དྭགས་ དྲྭ་ ཕྱྭ་ རྩྭ་ ཚྭ་ ཞྭ་ ཟྭ་ རྭ་ རྭི་ རྭང་ ལྭ་ ཤྭ་ སྭ་ བསྭ་ བསྭེ་ བསྭོ་ ཧྭ་ ཧྭག་ ཧྭགས་ ཧྭང་ 

Note that རྒྭ་ is not listed here due to being a proper noun.

\subsection{List of syllables containing the འུ suffix}

The following syllables can be found easily in many dictionaries:

ཀའུ་ ཀིའུ་ ཀེའུ་ ཀོའུ་ ཀྲའུ་ ཀྲུའུ་ ཁིའུ་ ཁེའུ་ ཁྱིའུ་ ཁྱེའུ་ ཁྲིའུ་ ཁྲུའུ་ ཁྲེའུ་ གའུ་ གོའུ་ གྲིའུ་ གྲེའུ་ གླེའུ་ འགིའུ་ འགོའུ་ རྒེའུ་ སྒའུ་ སྒེའུ་ སྒྱིའུ་ སྒྱེའུ་ སྒྲེའུ་ རྔེའུ་ སྔེའུ་ ཅེའུ་ ཅོའུ་ གཅིའུ་ གཅེའུ་ གཅོའུ་ ལྕེའུ་ རྗེའུ་ ཉེའུ་ སྙེའུ་ ཏེའུ་ གཏེའུ་ རྟའུ་ རྟེའུ་ སྟེའུ་ ཐའུ་ ཐིའུ་ ཐུའུ་ ཐེའུ་ ཐོའུ་ མཐེའུ་ དུའུ་ དེའུ་ དྲིའུ་ དྲེའུ་ མདེའུ་ རྡེའུ་ ལྡེའུ་ སྡེའུ་ ནའུ་ ནེའུ་ ནོའུ་ སྣེའུ་ དཔེའུ་ སྤའུ་ སྤེའུ་ སྤྱིའུ་ སྤྲེའུ་ ཕེའུ་ ཕྲའུ་ ཕྲེའུ་ འཕེའུ་ བེའུ་ བྱའུ་ བྱིའུ་ བྱེའུ་ བྲའུ་ བྲེའུ་ བྲོའུ་ འབེའུ་ སྦྲེའུ་ མིའུ་ མུའུ་ མོའུ་ མྱིའུ་ རྨེའུ་ སྨེའུ་ ཙིའུ་ ཙེའུ་ གཙེའུ་ རྩིའུ་ རྩེའུ་ ཚའུ་ ཚུའུ་ ཚེའུ་ མཚེའུ་ མཚེའུ་ རྫིའུ་ རྫེའུ་ གཞུའུ་ ཟེའུ་ ཡེའུ་ གཡིའུ་ རེའུ་ ལའུ་ ལིའུ་ ལེའུ་ ལོའུ་ ཤའུ་ ཤེའུ་ སིའུ་ སེའུ་ སྲིའུ་ སླེའུ་ བསེའུ་ ཧུའུ་ ཧེའུ་ ཧྲུའུ་ ཨའུ་

\bibliographystyle{plainnat}
\bibliography{tibliographyltx}
\end{document}
